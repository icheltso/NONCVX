\documentclass{article}
\newcommand{\dom}{\mathrm{dom}}
\usepackage{lineno}
\usepackage{grffile}

\linenumbers


\usepackage[utf8]{inputenc} % allow utf-8 input
\usepackage[T1]{fontenc}    % use 8-bit T1 fonts
\usepackage{hyperref}       % hyperlinks
\usepackage{url}            % simple URL typesetting
\usepackage{booktabs}       % professional-quality tables
\usepackage{amsfonts}       % blackboard math symbols
\usepackage{nicefrac}       % compact symbols for 1/2, etc.
\usepackage{microtype}      % microtypography

\usepackage{mystyle}
\usepackage{mathtools,booktabs,tabu,graphicx,hyperref,epstopdf,url,rotating}
%\usepackage{fullpage}
\usepackage{algorithm}
\usepackage{cite}
\usepackage{algorithmic}
\usepackage{fancyref}

\title{Non-smooth hyper-parameter learning }

\author{%
  Clarice Poon
}

\newcommand{\OO}{\mathbb{O}}

\newcommand{\commentout}[1]{}

\newcommand{\asinh}{\mathrm{arcsinh}}

\begin{document}

\maketitle


Consider
\begin{align*}
&\min_{y} h(y), \qwhereq h(y) = f(y, x(y)),\\
 &\qquad \text{s.t.}  \quad x(y)\in \argmin_x g(y,x).
\end{align*}
In general, we have
$$
\partial_x g(y,x(y)) = 0,
$$
and by the implicit function theorem, provided that $\partial_{x}^2 g(y,x(y))$ is invertible, $y\mapsto x(y)$ is differentiable with
$$
x'(y) = - \partial_{xx}g(y,x(y))^{-1} \partial_{yx} g(y, x(y)).
$$
One can then evaluate the gradient of $h$ as
\begin{equation}\label{eq:gradient_bilev}
\nabla h(y) = \partial_y f(y, x(y)) - \partial_x f(y,x(y))^\top  \partial_{xx}g(y,x(y))^{-1} \partial_{yx} g(y, x(y)).
\end{equation}


\section{Quasi-Newton?}
One approach is to  find $p_k$ such taht
$$
 \partial_{yx} g(y, x(y))\approx  \partial_{yy}g(y,x(y))  p_k
 $$
 and evaluate $\nabla h(y) \approx \partial_y f(y, x(y)) - \partial_x f(y,x(y))^\top p_k$.


Can we do a quasi-Newton approach?
We know that
$$
\partial_x g(y,x(y)) = 0
$$
Let $x_k\eqdef x(y_k)$. Then
\begin{align*}
\partial_x g(y_k,x(y_{k-1})) &=
\partial_x g(y_k,x_{k-1})- \partial_x g(y_k,x_k) \\
& = \partial_{xx} g(y_k, x_k) (x_{k-1} - x_{k}) + o(\norm{x_{k-1}-x_k})
\end{align*}
Suppose we find $B_{k}$ such that it minimises
$$
\min_B\norm{ B  \partial_x g(y_{k},x(y_{k-1})) - (x_{k-1} - x_{k}) }
$$
and treat it as an approximation to $\partial_{xx} g(y_{k}, x(y_{k})) ^{-1}$. The idea is to compute
$$\nabla h(y_k) \approx \partial_y f(y_k, x(y_k)) - \partial_x f(y_k,x(y_k))^\top  B_{k} \partial_{yx} g(y_k, x(y_k)).
$$

One possible update of $B_{k}$ is as $\tau_k\Id + u_{k} u_{k}^\top$. Define $s_k =x_{k-1} - x_k $ and $z_k  = \partial_x g(y_{k},x(y_{k-1}))$. We want to find diagonal + rank-1 matrix $B$ to minimise
$$
\min_B\norm{ B z_k  -s_k }
$$
\begin{itemize}
\item[i)] Define  $\tau_k = \dotp{s_k}{z_k}/\norm{z_k}^2$ and project onto $[\tau_{\min}, \tau_{\max}]$.  Note that before projection, $\tau_k = \argmin_\tau \norm{\tau z_k - s_k}$.
\item[ii)]  Let $B_0 \eqdef \gamma \tau_k \Id$ where $\gamma \in (0,1)$.
\item[iii)]  If $\dotp{s_k - B_0 z_k}{z_k}\leq 10^{-8} \norm{z_k}\norm{s-B_0 z_k}$ the $U_k=0$. Else: 
$$
U_k = \frac{(s_k - B_0 z_k)(s_k - B_0 z_k)^\top}{\dotp{s_k - B_0 z_k}{z_k}}.
$$
\item[iv)]  Let $B_k = B_0 + U_k$.
\end{itemize}
Note that for step iii) the choice of $U_k$ is precisely finding $U_k = uu^\top$ such that
$$
 B_0 z_k + u\dotp{u}{z_k} - s_k  = 0.
$$
\todo{
\begin{enumerate}
\item If we repeatedly updated $B_k$ with rank-1 matrices, show that $B_k$ converges to $\partial_{xx} g(y_* ,x(y_*))^{-1}$.
\item Suppose that $f = g$ and at iteration $k$, we have an approximate solution $\hat x_k \approx x(y_k)$. Possible ways of computing $\nabla h(y_k)$ include
\begin{itemize}
\item[i)] $p_1 = \partial_y f(y_k,\hat x_k))$
\item[ii)] $p_2 = \partial_y f(y_k,\hat x_k)) - \partial_x f(y_k, \hat x_k)^\top B_k \partial_{yx} f(y_k, \hat x_k)$
\item[iii)] $p_3 = \partial_y f(y_k, \hat x_k) + \partial_x f(y_k, \hat x_k)^\top  \partial_y \hat x_k$ where we obtain  $\partial_y \hat x_k$ via autodiff.
\end{itemize} 
The first and 3rd option have been analysed recently (Ablin et al). For the second approach, can we bound the difference between taking approximation $B_k$ and the true Hessian? 
\end{enumerate}
}





\subsection{When the outer and inner problems are the same}

Consider
$$
\min_y h(y), \qwhereq h(y) =  \min_x  f(x,y).
$$




By differentiating $\partial_x f(x(y),y) = 0$, we obtain  for $x = x(y)$, $$
\partial_x^2 f(x,y) x'(y)  =-\partial_x \partial_y f(x,y)
$$
and$$
\nabla h(x)  =   \partial_y f(x,y) - \partial_{xy} f(x,y)^\top \partial_x^2 f(x,y) ^{-1}  \partial_x f(x,y) 
$$
If $x(y) = \argmin_x f(x,y)$ is computed exactly, then $\nabla h(y) = \partial_y f(x,y)$. The question is what happens when $x(y)$ is only approximated by $\hat x$.  In this case, one can use the approximation
$$
\hat p = \partial_y f(\hat x,y) - \partial_{xy} f(\hat x,y)^\top \partial_x^2 f(\hat x,y) ^{-1}  \partial_x f(\hat x,y) 
$$
How effective is the quasi-Newton approximation to $\partial_x^2 f(\hat x,y) ^{-1} $?


Examples:
\begin{itemize}
\item[(i)]  Lasso $ f(x,y)\eqdef \frac{1}{2} \norm{x}^2 + \frac12 \norm{y}^2  + \frac{1}{2\la} L(xy)
$

\begin{align*}
\partial_y f= y +\frac{1}{\la} x\odot \nabla L(xy),\quad\partial_x f= x +\frac{1}{\la} y\odot \nabla L(xy)
\end{align*}

$$
\partial_{xy} f = \frac{1}{\la}\pa{ \diag(\nabla L(xy)) + \diag(y) \nabla^2 L(xy) \diag(x) }
$$
and
$$
\partial_{xx} f = \Id + \frac{1}{\la} \diag(y) \nabla^2 L(xy) \diag(y)
$$

\end{itemize}


\section{A differentiable approach to nonsmooth bilevel programming}


One example is where $g$ is nonsmooth is when $y$ correspond to a hyperparamter $\lambda$ and $x$ is the lasso regression coefficients:  $$
f(\lambda,\beta)\eqdef \norm{A_{\text{test}}\beta - y}^2\qandq g(\lambda,\beta) \eqdef \frac12 \norm{A_{\text{train}}\beta - b}_2^2 +\lambda \norm{\beta}_1.
$$

The difficulty is in this case is that since $g$ is non-smooth, the formula \eqref{eq:gradient_bilev} cannot be used. One alternative is to consider instead
$$
f(\lambda,(u,v))\eqdef \norm{A_{\text{test}} uv - y}^2\qandq g(\lambda,(u,v)) \eqdef  \norm{A_{\text{train}}uv - b}_2^2 +\lambda \norm{u}^2/2+\lambda\norm{v}^2/2.
$$
The advantage with this approach is that $g$ is a smooth function and one can show that the Hessian of $g$ is invertible when $\beta\eqdef u(\lambda)\odot v(\lambda)$ is a nondegenerate solution, that is,
$$
\max_{i\not\in\Supp(\beta)} \abs{ A_{\text{train}}^\top(A_{\text{train}}\beta - b)}_i <1.
$$

Things to do
\begin{itemize}
\item Check properties of the Hessian of $g$.
\item Acceleration using support pruning. 
\item Can we handle regularisers such ash $\norm{L\beta}_1$ where $L$ is a (possibly singular) linear operator?
\end{itemize}



For $$
\min_\lambda f(\lambda,\beta)
$$
where $\beta \in \argmin_\beta  \lambda\norm{L\beta}_1 + \frac12 \norm{A\beta- y}^2$,
consider instead
$$
\min_\lambda h(\lambda)\eqdef f(\lambda,\beta(v(\lambda),\lambda))
$$
where
$$
v(\lambda) \in \argmin_v \psi(v,\lambda)\eqdef \min_{\beta} \max_\alpha -\frac{1}{2\la}\norm{\al v}^2 + \frac{\la}{2}\norm{v}^2 + \frac12 \norm{A\beta - y}^2+\dotp{ L\beta }{\alpha}
$$
and
$$
\beta(v,\lambda) \in \argmin_{\beta}  \max_\alpha  \frac{1}{2\la}\norm{\al v}^2 + \frac{\la}{2}\norm{v}^2 + \frac12 \norm{A\beta - y}^2+\dotp{ L\beta }{\alpha}
$$
We certainly have that $\psi$ is differentiable and can compute $\nabla \psi$. To compute $\nabla h(\lambda)$, we need
$$
\partial_\la f + \partial_\beta f [\partial_v \beta \partial_\la v + \partial_\la \beta]
$$

\section{The square root lasso}
The square root lasso is
$$
\min_{\beta \in \RR^n} \norm{X\beta - y}_2 + \lambda \norm{\beta}_1.
$$
One interesting aspect of this is that when $y = X\beta_0 + w$, the minimiser $\beta$ satisfies
$$
\norm{\beta -\beta_0} \lesssim \norm{w}
$$
for some constant $\lambda$. This is remarkable since, for the Lasso, to achieve this kind of error bound, one would require that $\lambda\sim \norm{w}$ and some knowledge of the noise level is required.

One remark is that the square root lasso is equivalent to
$$
\min_{\sigma>0} \min_{\beta}\frac{1}{2\sigma} \norm{X\beta-y}_2^2 + \frac{\sigma}{2} + \lambda \norm{\beta}_1,
$$
and we can therefore write this in the bilevel formulation with
$$f(\sigma, \beta) = \frac{1}{2\sigma} \norm{X\beta-y}_2^2 + \frac{\sigma}{2} + \lambda \norm{\beta}_1$$
and
$$g(\sigma,\beta) = \frac{1}{2\sigma} \norm{X\beta-y}_2^2 + \lambda \norm{\beta}_1.$$

One question I have is what happens when we consider
\begin{align*}
f(\sigma,\beta) &=  \frac{1}{2\sigma} \norm{A_{\mathrm{test}}\beta-y}_2^2 +  \frac{\epsilon\sigma}{2} +\iota_{\sigma>0}\\
g(\sigma,\beta) &= \frac{1}{2\sigma} \norm{A_{\mathrm{train}} \beta-y}_2^2 +  \norm{\beta}_1 
\end{align*}
This is precisely the hyperparameter learning framework but with added regularisation on the parameter $\sigma$. Note that the outer problem can be written as an unconstrained smooth problem as follows:
Let $z = \sqrt{\sigma}$ and $v =  A_{\mathrm{test}}\beta-y$, then
\begin{align*}
\min_{z\in\RR}f(z, \beta(z^2)), \qwhereq &f(z,\beta)\eqdef\min_{ zv =  A_{\mathrm{test}}\beta-y} \frac12  \norm{v}^2 +\frac{\epsilon}{2} z^2 ,\\
&
\beta(z^2) \eqdef \argmin_\beta g(z^2,\beta)
\end{align*}
Notice that the minimisation problem in $f$ is convex wrt $v$, so by taking the convex dual, 
\begin{align*}
f(z,\beta) = \max_{\alpha\in\RR^m} - \frac{\epsilon}{2}  z^2 \norm{\alpha}^2 +\frac{\epsilon}{2}  z^2 +\dotp{\alpha}{ -  A_{\mathrm{test}}\beta +y}
\end{align*}
The maximiser $\alpha$ is unique (as the problem is strongly concave) and $$\partial_\beta f = -A_{\mathrm{test}}^\top \alpha \qandq \partial_z f = z\norm{\alpha}^2.$$
Numerically, we can certainly handle this, the question is whether this kind of regularisation is interesting in practice.

\newcommand{\Ate}{A_{\mathrm{test}}}
\newcommand{\yte}{y_{\mathrm{test}}}

\newcommand{\Atr}{A_{\mathrm{train}}}
\newcommand{\ytr}{y_{\mathrm{train}}}

Let $F(\sigma) = f(\sigma, \beta(\sigma))$.
Let's look at the optimality conditions\begin{align*}
\partial_\sigma f = \frac{-1}{\sigma^2} \norm{\Ate\beta -\yte}^2 + \frac{\epsilon}{2} \\
\partial_\beta f =  \frac{1}{\sigma} \Ate^\top (\Ate\beta -\yte) 
\end{align*}
Also, $\beta = \beta(\sigma)$ satisfies
$$
\Atr^\top \Atr \beta  = \Atr^\top  y -\sigma \sign(\beta)
$$
In general, $\sigma\mapsto \beta(\sigma)$ is differentiable almost everywhere with gradient
$$
 \beta'(\sigma)  = - (\Atr^\top \Atr)_{J,J}^{-1}  \sign(\beta).
$$
where $J=\Supp(\beta)$. So, when $F'(\sigma) = 0$, we have
$$
\frac{-1}{\sigma^2} \norm{\Ate\beta -\yte}^2 - \frac{1}{\sigma} \dotp{ (\Ate\beta -\yte) }{\Ate (\Atr^\top \Atr)_{J,J}^{-1}  \sign(\beta)}  + \frac{\epsilon}{2}   = 0
$$
which implies $\lambda = 1/\sigma$ satisfies, for $C\eqdef  \dotp{ (\Ate\beta -\yte) }{\Ate (\Atr^\top \Atr)_{J,J}^{-1}  \sign(\beta)}$,
$$
\lambda =  \frac{-C + \sqrt{C^2 + 2 \epsilon \norm{\Ate\beta -\yte}^2    }}{2\norm{\Ate\beta -\yte}^2 }
$$

NB: For the standard problem where $f(\sigma,\beta) =\frac12 \norm{\Ate\beta - y}^2$, then
$$ 
F'(\sigma) = -\dotp{(\Ate\beta - \yte)}{\Ate (\Atr^\top \Atr)_{J,J}^{-1} \sign(\beta) } .
$$
Suppose $\Ate = \Atr$, then this says that
$$
F'(\sigma) = \sigma\dotp{ (\Atr^\top \Atr)_{J,J}^{-1} \sign(\beta)}{\sign(\beta)} > 0
$$
which means that we optimise to $\sigma = 0$ as expected.
%Note that
%\begin{align*}
%&\Ate^\top (\Ate \beta - y) - \Atr^\top (\Atr \beta -y)\\
%&= (\Ate - \Atr)^\top (\Ate \beta - y) +  \Atr^\top (\Ate - \Atr) \beta 
%\end{align*}
%So, provided that $\norm{(\Ate - \Atr)}_{1\to 2} = \Oo(\delta)$,
%\begin{align*}
%&\norm{\Ate^\top (\Ate \beta - y) - \sigma \sign(\beta)}_\infty\\
%&= \norm{(\Ate - \Atr)^\top}_{2\to\infty} \norm{ (\Ate \beta - y) }_2+  \norm{\Atr^\top}_{2\to\infty} \norm{(\Ate - \Atr)}_{1\to 2} \norm{ \beta }_1 = \Oo(\delta).
%\end{align*}
\end{document}